When considering visual encodings for uncertainty it is imporant to not
only consider the perceptual accuracy of the encoding but also how the
user will interpret this uncertainty.

\begin{itemize}
\tightlist
\item
  it is important when designing visual encodings of uncertainty to
  understand the user
\item
  right now much of this is handled on a case by case basis or not
  identified
\item
  why are different tools developed for similar tasks?
\end{itemize}

\ttwnote{important to say who's decisions we are trying to model}

Concepts such as personality factors \citep{ref} and cognitive biases
\citep{ref} have previously been discussed by the visualization
community. However, these factors are designed to be fixed for life.
Problem solving heuristics, on the other hand, differ on a case by case
basis. The same user may choose different heuristics for the same tasks
and two different users may choose different heuristics for the same
task.

\begin{itemize}
\tightlist
\item
  it's important for the tool designer to understand the final decision
  making process that the tool will use is what's important
\item
  don't consider it formally or the alternative heuristics/strategies
\end{itemize}

For example, we ran a user study with two interfaces. The designer of
the system was held constant and the two interfaces only differed by the
addition of a small whisker (see Fig. ???). This was an open-ended
investment task and participants were asked to report on their
confidence in using the interfaces as well as their interface
preference. Based on the interview we conducted with participants, we
found that they used two different strategies to make their decision
about which risk/return level was acceptable. When divided into the two
groups one group (the sweet spotters) showed a strong preference for one
interface.

\begin{itemize}
\tightlist
\item
  need a transition here to the benefits of this for vis
\end{itemize}

In summary our contributions are:

\begin{itemize}
\tightlist
\item
  we introduce problem solving heuristics
\item
  map these to previous design study work on uncertainty/parameter space
  exploration
\item
  discussion for implications for design allowing visualization
  designers to design better (more easily adopted) visualization tools
  faster (fewer iterations)
\end{itemize}

\ttwnote{another benefit is that these methods have methodologies for 
identifying them and studying them}

\ttwnote{can this be used to better identify a
successful optimization design study?}

\section{Problem solving strategies}\label{problem-solving-strategies}

\subsection{Overview}\label{overview}

\begin{itemize}
\tightlist
\item
  difference between ``expert'' decision makers and general decisions
\item
  general decisions are more common in vis
\item
  \emph{definition}: selecting the best candidate amongst a number of
  alternatives where the candidates have a number of competing
  objectives
\end{itemize}

\subsection{Heuristics}\label{heuristics}

from the adaptive decision maker:

\subsubsection{Weighted additive rule}\label{weighted-additive-rule}

\begin{itemize}
\tightlist
\item
  Qualitative final comparison
\item
  weight all objectives by how important they are
\item
  the weighted sum determines the ``best'' choice
\end{itemize}

\subsection{Equal weight}\label{equal-weight}

\begin{itemize}
\tightlist
\item
  subset of weighted additive rule
\item
  all objectives are treated equally
\end{itemize}

\subsection{Satisficing}\label{satisficing}

\begin{itemize}
\tightlist
\item
  the first acceptable alternative that is encoutered is selected
\item
  acceptable is having factors above the acceptable value
\end{itemize}

\subsection{Lexicographic}\label{lexicographic}

\begin{itemize}
\tightlist
\item
  sort by most important attribute
\item
  ties are broken by further attributes in order of importance
\item
  the difference to satisficing is that in lexicographic \emph{all}
  alternatives are considered, not just the first acceptable one
\end{itemize}

\subsection{Elimination by aspects}\label{elimination-by-aspects}

\begin{itemize}
\tightlist
\item
  the most important attribute is selected along with a ``minimum
  acceptable value''
\item
  candidates with values below this are eliminated
\item
  the next round of filtering starts with the second most important
  attribute
\end{itemize}

\subsection{Majority of confirming
dimensions}\label{majority-of-confirming-dimensions}

\begin{itemize}
\tightlist
\item
  pairwise comparison (major distinction)
\item
  number of aspects that are ``better'' in the pair means it wins
\item
  loser is eliminated from consideration
\item
  winner is compared against the next candidate
\end{itemize}

\subsection{Frequency of good/bad
features}\label{frequency-of-goodbad-features}

\begin{itemize}
\tightlist
\item
  number of positive aspects minus number of negative aspects
\end{itemize}

\section{Design study methodology papers
tags}\label{design-study-methodology-papers-tags}

To see if users indeed use different strategies we coded the 21 papers
from the visual parameter space analysis paper \citep{Sedlmair:2014}
with the decision heuristic types above (or none). We found that 6/8 of
the papers described some sort of heuristic when describing the task
analysis, user characterization, or case study.

\section{Discussion}\label{discussion}

Some interesting facts so far:

\subsection{Most popular are sorting or
elimination}\label{most-popular-are-sorting-or-elimination}

Lexicographic and elimination by aspects are the most popular heuristics
by far. Both of these concentrate on one parameter at a time and are
about filtering alternatives.

\subsection{Satisficing?}\label{satisficing-1}

\begin{itemize}
\tightlist
\item
  not sure if this fits with the goals of vis

  \begin{itemize}
  \tightlist
  \item
    vis is about making analytic decisions
  \item
    this is more something to watch out for\ldots{} cognitive bias?
  \end{itemize}
\item
  is this ever a good strategy/heuristic? \ttwnote{research this}
\end{itemize}

\subsection{Weighted parameter heuristics are
missing}\label{weighted-parameter-heuristics-are-missing}

This is supported by papers such as LineUp \citep{Gratzl:2013} but none
of the tools in the vPSA paper work this way
\ttwwarning{I think...}

\subsection{The decision strategy is never discussed or
compared}\label{the-decision-strategy-is-never-discussed-or-compared}

in the design study papers and in the VPSA paper:

\begin{itemize}
\tightlist
\item
  alternatives never considered
\item
  the word ``decision'' is never defined or losely defined
\end{itemize}

\section{Conclusion}\label{conclusion}

yay!

\section{Other papers}\label{other-papers}

\begin{itemize}
\tightlist
\item
  \citep{Berger:2010}
\item
  \citep{Gratzl:2013}
\end{itemize}
