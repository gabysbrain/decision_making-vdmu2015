
The current literature on uncertainty visualization follows two main streams: perceptual studies and design studies.
\emph{Perceptual studies} such as MacEachren et al.~\citep{MacEachren:2012} 
help us to better understand 
how simple, low-level uncertainty tasks can be supported by visual encodings. 
While the results of such studies can be seen as generalizable across a wide 
range of humans, the approach is not accounting for higher-level aspects, such as cognitive skills,
problem solving strategies, or decision making.

With these higher goals in mind, there has been an increasing focus on \emph{design studies}~\citep{Sedlmair:2012}.
In this qualitative approach, researchers work together with end users to better understand their 
(ill-specified) data analysis tasks, and to create a visualization tool
that supports these. In doing so, design studies provide us with instances of how
actual problems were broken down into design decisions of actual tools, going far beyond
low-level perceptual reasoning. These instances, however, are highly problem- and domain-specific.
Therefore, while some of the knowledge gained in designing these tools is
``transferable'' to other visualization tools, the results are often not
``generalizable'' to the larger populace.

In this position paper, we argue that there is a gap between these two lines of work.
While there are general task taxonomies~\citep{Brehmer:2013,Shneiderman:1996} and 
models~\citep{Tory:2004,Sedlmair:2014}
attempting to bridge the gap, we argue that a more systematic \emph{user modeling} is missing.
The visualization community has already started examining personality types and
how those correspond to visualizations~\citep{Conati:2014}. Beyond that, we advocate for a more thorough 
investigation and understanding of the overall \emph{problem solving} strategies when extracting information from data, 
and the ultimate \emph{decision making} processes.
While these terms are not new to the visualization community, the 
literature from other domains such as Complex Problem Solving (CPS)~\citep{Frensch:2005}
and Decision Theory~\citep{Payne:1993} suggests that there is much more to say about these characteristics,
than what we currently discuss in our community. For instance, people are known to have different strategies to go about decision making, filtering information,
and adapt strategies based on, among other things, the number of attributes 
to consider, time pressure, and the mode of information 
display~\cite{Payne:1993}.
We deem a better understanding of such aspects specifically important for uncertainty visualization.
Not only could they help to run better perceptual studies by directly accounting for user characteristic variables 
(e.g., users following a certain strategy are ``better'' performing than others), but also support design study researchers
by providing a better framework for problem and task characterization.

In this position paper, we seek to conduct some first steps towards such a more
sophisticated understanding of the role of users in (uncertainty)
visualization. Toward this gaol, we make the following contributions:
\begin{itemize} 
\item We introduce 6 different decision making strategies from
      the decision theory literature~\citep{Payne:1993}.
\item We analyze a set of 21 design study papers, classifying them according 
      to these strategies.  
\item We discuss and advocate the following questions: What
      are common decision making strategies? How can current approaches of task
      taxonomies and mantras be reconciled under the lens of decisions making?
      How can we better support \emph{different} decision making strategies in
      our tools?  
\end{itemize}
  



%Most prominent forms of evaluation of uncertainty visualization systems 
%are through one of two distinct fronts.
%On one side are empirical studies while on the other side are design studies.
%\tmnote{just about all the vis research is empirical. design studies are empirical? you are mixing sth here?}
%The designs of uncertainty visualizations must not only take into account 
%perceptual accuracy of the visualization but also how that uncertainty will
%be interpreted and how that interpretation will affect decisions made based
%on the chosen representation. \tmnote{if that is the statement, you can start this with that most work on uncertainty vis has focused on the perceptual aspects. however, we don't know how it actually figures in in solving tasks (which is traditionally design study land).} In this proposal,
%we argue that a more holistic approach to evaluation and design is required
%by evaluating users based on and incorporating decision making heuristics
%while designing visualization systems. 
%%\tmnote{"Vis research" is a bit big here. there is lots more (like technique
%%papers, etc.)}
%%\msnote{and field studies? what else?}
%
%\ttwnote{probably want to cut down on the user and design study paras}
%Lab studies concentrate mostly on perceptual aspects which are low-level visual
%understanding and are considered to be more or less fixed across humans
%(barring exceptions like color-blindness). Furthermore, the testing
%methodologies for studies such as these are relatively mature with fixed
%objectives and protocols. However, what they can measure is limited but highly
%generalizable. Most studies done this way deal with preattentive 
%processing~\citep{McGrath:1995} or
%very low-level tasks. Because of this, visualizations tested in this way must be
%boiled down to their raw elements and even adding small interactions can make
%getting useful results difficult.  
%%Furthermore, it is not always clear exactly how to apply the results of 
%% these studies in when designing visualization 
%%tools \tmnote{disagree} \msnote{why? example}. 
%Solving higher-level tasks, however, requires cognitive 
%skills that can be different among people and that require a different 
%methodology to study/analyse and support~\citep{Frensch:2005}.
%
%On the other hand, design studies are highly focused on a particular
%application scenario~\citep{Sedlmair:2012}. While not as generalizable, the end goal of a design
%study is a tool that is designed to be used in a real-world application
%scenario. In addition to selecting proper visual encodings that are
%perceptually accurate, the integration of multiple views as well as
%interactions between these views must be taken into account
%\tmnote{perceptually accurate -> talk about expressiveness and effectiveness
%from Munzner 2014}. In the course of
%designing one of these tools one must also carefully consider what tasks the
%intended users want to accomplish as well as how they will employ these tasks
%to complete their analysis goal. However, the feedback in design studies is
%necessarily qualitative and limited. A new visualization tool involves changing
%many variables and any effect seen as a result of a focused lab study
%(for example) could be attributed to any combination of these changes.
%Therefore, while some of the knowledge gained in designing these tools is
%\emph{transferable} to other visualization tools, the results are often not
%\emph{generalizable} to the larger populace~\ttwwarning{ref from Michael}.
%We hypothesize that this is because different
%users have different problem solving strategies. 
%\ttwnote{point out shortcomings of Gestalt theory?}
%
%There is still a large gap in research between the results from perceptual
%studies and design studies. Some researchers have attempted to bridge this gap
%with task taxonomies~\citep{Brehmer:2013,Shneiderman:1996} or 
%models~\citep{Meyer:2012,Sacha:2014}. \tmnote{unclear how meyer and sacha address this gap!? I would find it more appropiate to cite Tory+Moeller's 2004 paper since they introduced the idea of the mental model of the user for the first time afaik} These are largely focused on 
%``what'' users are doing and not
%``why'' they are doing it. Brehmer et al.\ do list ``why'' tasks but 
%\ttwnote{need to read bremer again to compare but their why tasks are somewhat vague...}.
%With the addition of
%a better modeling of the user from cognitive principles, namely,
%what is the overall problem solving process of the user in terms of solving the
%problem of extracting information from data? \tmnote{prev. not a sentence} By making user modeling more
%explicit we could shorten the number of iterations for design studies since we
%would have a better understanding beforehand what visual encodings and
%interactions will help the user accomplish their goal beyond perceptual
%understanding \msnote{vague...}
%\ttwnote{not sure why vague, this seems really obvious to me}. 
%In addition, better user modeling will reduce the number of
%``unknown independent variables'' when conducting a user study and will lead to
%being able to run lab studies on much more complex visual encodings and
%interactions and reason about the results better (and is another independent
%variable) \tmnote{unclear}
%\ttwnote{this also seems obvious to me, not sure how to clean up}.
%\tmnote{unclear why it will be better?! better as in more 'accurate' or 'faster' or 'prettier' or 'easier' for domain scientists to do themselves or ...}
%
%%For example, we ran a lab study with two different interfaces in order
%%to test if a simple visual encoding of the sensitivity of a parameter to
%%a simulation model will make a more effective visualization. One issue
%%with this system is the relatively open-ended nature of the task: the
%%user could select any reasonable value for the expected return and still
%%complete the study. Before looking at the results we found that the
%%participants in the study could be divided into two different groups
%%based on their strategy in selecting their portfolio. Based on the
%%strategy we we able to find that the ``sweet spotters'' had a
%%statistically significant preference for the SA interface over the
%%other, while the ``risk fixer'' group showed no clear interface
%%preference. We did not anticipate the two different groups when
%%designing the study (we thought everyone would be a ``sweet spotter'')
%%but knowing that there would be a way to classify the users beforehand
%%would have beeen helpful. With better user characterization this would
%%have been possible.
%
%The visualization community has already started examining personality types and
%how those correspond to visualizations~\citep{Conati:2014}. While this does
%make the visualization choice more ``personal'', personality types are designed
%to be consistent over time and therefore cannot account for the case where the
%same user will use different problem solving strategies depending on the task.
%We argue for a more situation-based approach to modelling a user interacting
%with a visualization tool. We propose to look at the entire visual exploration
%process as a case of solving a complex problem. Complex problem solving (CPS) can be
%defined as: ``CPS occurs to overcome barriers between a given state and a
%desired goal state by means of behavioral and/or cognitive, multi-step
%activities''~\citep{Frensch:2005}. One of the major goals of a visualization
%system is insight which has been defined~\citep{Yi:2008a} in many ways but all
%definitions are remarkably similar to the definition from Mayer for learning
%which is ``information aquisition'' and ``knowledge
%construction''~\citep{Mayer:2009}. Even with these definitions though there is
%still the question of how to measure insights in a visualization system. We
%feel that the methods from e-learning research can be used to measure the
%aquisition of insights \tmnote{sharpen} 
%\ttwnote{maybe drop the whole 2nd half of this para and just reference it 
%really quick in the talk}. \tmnote{yes, but you need to explain what you mean with 'situation based approach'!}
%
%Similarly, one of the reasons for visualizing the uncertainty of a simulation
%model in the first place is for making actionable decisions. With these
%decisions one often has a set of choices with multiple competing objectives.
%The work done by researchers in decision theory specifically addresses the
%methods by which people make contingent decisions in which, all alternative
%action possibilities, events that relate actions to outcomes, and the objective
%values are all available to the decision maker~\citep{Payne:1993}.  \tmnote{the next sentence doesn't bring anything to the table, delete:We should
%look into this work that researchers have done on decision making where they
%have various heuristics how people go about making decisions. }
%
%In this proposal we begin the overall examination of this process \tmnote{what is 'this process' referring to?} and its
%relevance and benefits to visualization research by first looking into the
%decision making heuristics but later we plan to examine the entire model.  \tmnote{'entire model' of what?} As a
%first step, we examine how current visualization systems employ the 
%various decision making strategies
%and implications for design of visualization
%tools.  We would like to introduce the
%following questions for discussion: \tmnote{not a good question, delete:} is visualization already doing this, what
%types of decision making strategies have been supported \tmnote{... in the vis community/ in vis tools}, the common decision
%making strategies in visualization systems \tmnote{not a question}, and should \tmnote{not 'should we steer' more 'how to support'?} we be steering users to
%certain decision making strategies.
%The goal of these questions is to discuss how incorporating decision making
%heuristics will improve the development and evaluation of visualization tools.
%
%\tmnote{you really have to cut down on motivation, can't be more than 2 pages text, a 3rd page for refs ok}

\section{Decision Making Strategies}\label{heuristics}

In the following, we enumerate 6 different decision making strategies from the book \emph{``The adaptive decision
maker''}~\citep{Payne:1993}, and discuss them in the light of visualization research.
The strategies characterize how people make contingent decisions in which, all alternative
action possibilities, events that relate actions to outcomes, and the objective
values are all available to the decision maker.
All strategies assume that the decision maker (or user in our case)
is selecting a decision item from a list of discrete choices. Each item
has multiple attributes (objectives). These objectives are competing with
each other in the sense that there is no clear optimal choice. 
%What is also
%important about the strategies is that depending on how information is 
%presented to participants different decisions strategies will be employed
%and different optimal selections will be made~\citep{Jarvenpaa:1990}.

%All these heuristics assume that the decision maker (or user in this case)
%is selecting a decision item from a list of discrete choices. Each item
%has multiple attributes (objectives). These objectives are competing with
%each other in the sense that there is no clear optimal choice. What is also
%important about the heuristics is that depending on how information is 
%presented to participants different decisions heuristics will be employed
%and different optimal selections will be made~\citep{Jarvenpaa:1990}.
%The following heuristics \tmnote{need a proper reference to the table} are all described in \emph{The adaptive decision
%maker}~\citep{Payne:1993}.

To better understand how these different strategies are supported by current visualization tools,
we coded 21 design study papers on visual parameter space analysis tools
(the same set as Sedlmair et al.~\citep{Sedlmair:2014}), and classified them according to strategy 
type. We found that $15/21$ of the papers described
some sort of strategy when describing the task analysis, user
characterization, or case study. Table~\ref{tbl:decision-types} shows an overview of the results;
specific examples will be discussed below.
%%MS: This is a result, i.e. should later, I moved it
%Lexicographic and elimination by aspects are the most popular heuristics
%by far. Both of these concentrate on one parameter at a time and are
%about filtering alternatives. 

\begin{table}[tb]
  \begin{center}
    \caption{We classified the 21 papers listed in 
             Sedlmair et al.~\citep{Sedlmair:2014} for which decision making
             heuristics, if any, are employed by the users of the tools
             when describing the task analysis, user characterization,
             or case study.  We could identify one of the decision making 
             heuristics in $15/21$ of the papers. The lexicographic and 
             elimination by aspects heuristics were the most commonly
             employed.
    }
    \label{tbl:decision-types}
    \inputfile{paper_codes}
  \end{center}
\end{table}



%\subsection{Experts in expert %situations}
%\label{experts-in-expert-situations}

\subsection{Strategy 1: Weighted additive rule}\label{weighted-additive-rule}

The weighted additive rule considers all
attributes of the various choices at once and directly considers 
the weights. The user selects weights for the various attributes and 
the option with the highest weighted sum of attributes.
A version of this strategy considers all attributes evenly but still
the selection is made based on the total weighted value.

Practical issues with this include mapping attributes to numerical values
and examining the effects of adjusting the weights. In our evaluation of
design studies this rule is not directly supported. We considered that
systems supporting the evaluation of distance functions such as 
in systems such as ParaGlide~\citep{Bergner:2013} as using this strategy.
LineUp~\citep{Gratzl:2013} is the only system as far as we know that
directly supports the evaluation of numerical weights on objectives. 
\tmnote{But Tuner does too.}
\ttwnote{not really, how?}

\subsection{Strategy 2: Lexicographic}\label{lexicographic}

Rather than considering all attributes at once the user can instead simply
order the selections by the most important attribute to them. Ties within this
attribute are broken by sorting by the second most important attribute and so
on until a clear candidate appears at the top. This process has been named
lexicographic by the decision making community.

Among our 21 reviewed papers, this is a relatively common strategy.
Many visualization systems, for example, Tuner~\cite{Torsney-Weir:2011} and
Luboschick et al.~\citep{Luboschik:2014} allow the user to prioritize a few
objective measures and evaluate the outputs based on those few. Naturally,
when presented with more attributes than can be easily compared between
one can filter
these attributes down to just the core set. This is a form of dimension 
reduction (filtering).

\subsection{Strategy 3: Elimination by aspects}\label{elimination-by-aspects}

Rather than concentrating on finding the optimal decision, one could
instead set thresholds and filter out the unacceptable decisions. This process
is repeated until a final choice remains. This process is called
elimination by aspects.

This heuristic is frequently employed in many visualization systems with
perhaps the system developed by Spence et al.~\citep{Spence:1995} being a
direct application of this strategy. The user can interactively filter
objective (output) dimensions into acceptable and unacceptable regions and
see that effect on the selections in the parameter space. 
Vismon~\citep{Booshehrian:2012} also uses a similar strategy.

\subsection{Strategy 4: Frequency of good/bad features}
\label{frequency-of-goodbad-features}

If, like in the elimination by aspects strategy, one can articulate for
each attribute a good and bad level then one could label each attribute
of each decision option by this. Then one could simply count the number of
attributes that are labeled "good" and compare this against the number of
"bad" labels and select the option with the highest difference. This has
been called the frequency of good and bad features.

This strategy is never directly addressed by the visualization community
but the closest example we could find was the work by 
Coffey et al.~\citep{Coffey:2013}. In their system the user could browse
through simulations and find similar ones which may be closer to what
the user is looking for. The frequency of good and bad features between the
choices of needle design determined the final needle design decision.

\subsection{Strategy 5: Satisficing}\label{satisficing}

Satisficing does not consider the entire set of decision options holistically.
Rather, when employing this strategy, one considers each option one at a time.
Whether all attributes of the choice are considered or just a few depends on
the person but either way the first acceptable option encountered is selected.
We did not see any evidence in the literature of this strategy being
employed. In some ways it is at odds with visualization tool development.
Usually, one assumes that all data must be examined or all data is known
up-front.
%Selecting the number of samples to take from a simulation, however, may suffer
%from this since the choice of if sufficient samples have been taken is never
%evaluated in systems such as Tuner~\citep{Torsney-Weir:2011} or
%Paramorama~\citep{Pretorius:2011}. \ttwnote{maybe delete this...}

\subsection{Strategy 6: Majority of confirming
dimensions}\label{majority-of-confirming-dimensions}

As an alternative to looking at all decision options at once one could 
instead examine them in a pairwise manner. The "winner" of each pairing is
compared against the next choice and so on until a final optimal choice
prevails.

This is also not very common in visualization perhaps because the pairwise
comparison is quite labor intensive. Also, like satisficing, depending on the
order that the options are considered one may get a very different outcome. In
Fluid Explorer~\cite{Bruckner:2010} and Paramorama~\citep{Pretorius:2011} this
is done by the user manually selecting an optimal candidate simulation from 
a list of simulation outputs.
%\ttwnote{Eric Brochu's work is probably the best example} \tmnote{yes and no, it is an active learning approach / feature, but I do not get to see all comparisons, just the ones the algorithm needs me to look at.}

%\subsection{Summary}

\section{Questions}\label{questions}

Based on this analysis, we would like to address a number of questions during the workshop.

\subsection{What are common decision making strategies?}

Lexicographic and elimination by aspects are the most popular heuristics
by far. Both of these concentrate on one parameter at a time and are
about filtering alternatives.

Only one of the papers examined in Sedlmair et al.~\cite{Sedlmair:2014},
ParaGlide~\citep{Bergner:2013}, concentrated on a
weighted strategy. ParaGlide did this though computing and
investigating a distance metric rather than directly adjusting the
weights. We're not sure why this is not a very common strategy in the
tools we examined. As far as we know, there is only one tool published,
LineUp~\citep{Gratzl:2013}, that allows the user to directly see the
effect of manipulating weights.
\ttwnote{what else...}

%It's not very clear why this is not so common perhaps just momentum? Is
%this hidden in other visualization systems under the monkier of
%``distance metric''? Also, weighting objectives is a relatively
%numerical process and perhaps this is simply done without visualization
%support where, for example, numerical optimization would be very
%effective?

%\tmnote{the ValueCharts paper did this. Further, I do believe it is part of some systems, but I couldn't point you to them now.}


\subsection{How can current approaches of task taxonomies and mantras be reconciled under the lens of decision making?}

%One question is do these problem solving heuristics exist already in the
%visualization literature but just in a different name? From the literature
%search we conducted the answer is no. There are elements of task support for
%these heuristics, like filtering, but a comprehensive strategy is never
%identified. 

Visualization systems are often designed around "overview first, zoom and
filter, details on demand"~\cite{Shneiderman:1996}, a "global to local,"
or "local to global" search strategy~\cite{Sedlmair:2014}. Both task 
taxonomies and decision making heuristics are developed around how users
go about solving tasks.
Task taxonomies
are focused on the engineering aspects of the tools in terms of what
features are required, for example, filtering or zooming. The decision
making heuristics are more focused on how these tasks can be combined to
solve a problem.
With better understanding of the decision making process a user will go
about one could better decide which tasks to prioritize for which users.
We propose that linking these task taxonomies with the different problem 
solving heuristics in terms of which heuristics are supported by which 
tasks will allow for more robust decisions about which tasks, and by
extension, which visual encodings to employ when designing a visualization
tool.

%These may help to
%address the more global heuristics like weighted additive or lexicographic.
%However, the more comparative heuristics such as the majority of confirming 
%dimensions are much more local in nature, involving the direct comparison
%of multiple options.  In some ways this is the local to global
%strategy identified in Sedlmair et al.~\citep{Sedlmair:2014}. The user wants
%to start with some known value and then gradually expand their seach from
%this single focus point.

%\tmnote{Needs some thought. Rather, how can the current approaches of task taxonomies and mantras be reconciled with work in the problem solving / decision making world. What are the similarities, what are the differences?}
\ttwnote{still rough...}


\subsection{How can we better support \emph{different} decision making strategies?}

Visualization concentrates on information
presentation. There is evidence that the presentation of information will
influence which decision making strategy one employs~\citep{Jarvenpaa:1990}.
On the other hand, in real-world decision making there is often no clear
optimal decision to be made so it is difficult to evaluate which decision
making heuristic is best. However, it seems that it is best to at least
comprehensively examine all the options.

One of the strengths of a visualization tool is the ability to switch modes
dynamically during data exploration. One could envision combining a
global overview in combination with a focused pairwise comparison system.
The global overview would remind the user of how much of the \emph{global}
space they have explored. In addition, one could also dynamically filter
the remaining options to reduce the number of comparisons in a process
not unlike active learning.

%\subsection{Common decisions in vis systems}
%\label{common-decisions-in-vis-systems}

%Based on our literature survey we found certain areas of parameter space
%exploration where a decision is being made.  For example, is the sampling
%sufficient, what model to select, and what parameter settings are good?

%\tmnote{this section is unclear to me.}

%\subsection{Weighted parameter lists are
%missing}\label{weighted-parameter-lists-are-missing}

\section{Conclusion}

We are proposing to study more closely the impact that uncertainty information
has on decisions and the implications for visual tool design. 
We have enumerated some common decision making heuristics and evaluated
if these are currently addressed by visualization systems today. We found that
these heuristics are not directly addressed by current systems and offered up
several questions for discussion going forward.

