%% These three lines bring in essential packages: ``mathptmx'' for Type 1
%% typefaces, ``graphicx'' for inclusion of EPS figures. and ``times''
%% for proper handling of the times font family.

\usepackage{tikz}
\usepackage{mathptmx}
\usepackage{amsmath}
\usepackage{graphicx}
\usepackage{times}
\usepackage{subcaption}
\usepackage{supertabular,booktabs}
\usepackage{enumitem}

%% maxwidth is the original width if it's less than linewidth
%% otherwise use linewidth (to make sure the graphics do not exceed the margin)
\makeatletter
\def\maxwidth{%
  \ifdim\Gin@nat@width>\linewidth
    \linewidth
  \else
    \Gin@nat@width
  \fi
}
\makeatother

%% natbib stuff
\usepackage[numbers]{natbib}

%% tightlist is an artifact from pandoc...
\providecommand{\tightlist}{%
  \setlength{\itemsep}{0pt}\setlength{\parskip}{0pt}}

%% We encourage the use of mathptmx for consistent usage of times font
%% throughout the proceedings. However, if you encounter conflicts
%% with other math-related packages, you may want to disable it.

%% This turns references into clickable hyperlinks.
\usepackage[bookmarks,backref=true,linkcolor=black]{hyperref} %,colorlinks
\hypersetup{
  pdfauthor = {},
  pdftitle = {},
  pdfsubject = {},
  pdfkeywords = {},
  colorlinks=true,
  linkcolor= black,
  citecolor= black,
  pageanchor=true,
  urlcolor = black,
  plainpages = false,
  linktocpage
}

% notes for everyone
% remove the draft argument to remove all comments and the index
% to check the paper size, submit, etc
\usepackage[draft,multiuser,marginclue,nomargin,inline,index]{fixme}
%\usepackage[multiuser,marginclue,nomargin,inline,index]{fixme}
%\usepackage[usenames,dvipsnames]{color}
\definecolor{ttwgreen}{RGB}{75,135,73}
\fxusetheme{colorsig}
\FXRegisterAuthor{tm}{atm}{\color{red}TM}
\FXRegisterAuthor{ttw}{attw}{\color{ttwgreen}TTW} % creates ttwnote, ttwwarning, ttwerror, ttwfatal commands
\FXRegisterAuthor{ms}{ams}{\color{blue}MS}

% comments in square brackets
\makeatletter
\renewcommand*\FXLayoutInline[3]{%
  {\@fxuseface{inline}\ignorespaces[#3 \fxnotename{#1}: #2]}}
\makeatother
